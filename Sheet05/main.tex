\documentclass{article}[a4paper]

\usepackage[english]{babel}

\usepackage{subcaption}
\usepackage{graphicx}

\usepackage{amsmath}
\usepackage{amsthm}
\usepackage{amsfonts}
\usepackage{tikz}
\usepackage[plain]{algorithm}
\usepackage{algpseudocode}

\usepackage[plain]{algorithm}
% For code snippets
\usepackage{listings}
\usepackage{color}

% Define colors for code
\definecolor{codegray}{rgb}{0.5,0.5,0.5}
\definecolor{codepurple}{rgb}{0.58,0,0.82}
\definecolor{backcolour}{rgb}{0.95,0.95,0.92}


\lstdefinestyle{mystyle}{
backgroundcolor=\color{backcolour},
commentstyle=\color{codegray},
keywordstyle=\color{blue},
numberstyle=\tiny\color{codegray},
stringstyle=\color{codepurple},
basicstyle=\ttfamily\footnotesize,
breakatwhitespace=false,         
breaklines=true,                 
captionpos=b,                    
keepspaces=true,                 
numbers=left,                    
numbersep=5pt,                  
showspaces=false,                
showstringspaces=false,
showtabs=false,                  
tabsize=4
}

\lstset{style=mystyle}


\title{
    \vspace{2in}
    \textmd{\textbf{Homework: Random Walk and Entropy}}\\
    \normalsize\vspace{0.1in}\small{Due on Friday 22.11 at 10:15am}\\
    \vspace{0.1in}
    \vspace{3in}
}

\author{Dominic Nieder, Awais Ahmed}
\date{\today}


\begin{document}

\section{Introduction}

In this exercise we study how a gaussian distribution follows from a simple one-dimensional random walk

\section{Random Walk}
\subsection{Implentation \& Results}

The Implentation of the random walk is simple and quick to implement. 
We take $N=20000$ as the total number of iteration steps and $number\_of\_particles=10000$. The positions get initaited with $x_i=0$ for all particles $i$. The variable $x$ contains all the positions for all iteration steps. There is also the variable $n_bins$ which is defined for the histogramms \ref{fig:probability_disti}. \\
Within the for-loop we generate an array of uniformly sampled random numbers in the variable $dw$, which get conditioned in the next step with the \texttt{np.where}-function and replaced with $-1$ or $1$ respectivly. Last but not least the the positions get updated with in respect to the previous positions and the randomly sampled numbers. T

\begin{lstlisting}[language=python]
N=20000 
n_bins=25
number_of_particles=10000  
x = np.zeros((N+1,number_of_particles),dtype=int) 
for i in range(N):
    dw= np.random.uniform(low=0.0,high=1.0,size=number_of_particles)  
    dw = np.where(dw<0.5, -1,1)  
    x[i+1,:]= x[i]+dw  # then the position get iterated
\end{lstlisting}

In the figures \ref{fig:probability_disti} we can four bin plots \ref{fig:first} \ref{fig:second} \ref{fig:third} and \ref{fig:fourth} that count positions in a one-dimensional space. Each showing the particle distribution of the same random walk for different iteration - counted with $25$ bins respectivly. Starting with the lowest iteration number in \ref{fig:first} and the most in \ref{fig:fourth} that has the largest iteration number. One can clrearly recognise how the gaussian bell-shaped distribution emerges from the randomwalk alone.  \\
The figures from \ref{fig:first} \ref{fig:fourth} show a tendency of becoming more smooth for larger total timesteps. This can especially be seen when considering the next set of plots in figure \ref{fig:statistics_of_RW}.

\begin{figure}[h]
\centering
\begin{subfigure}{0.4\textwidth}
    \includegraphics[width=\textwidth]{RW_N100.png}
    \caption{$N=100$}
    \label{fig:first}
\end{subfigure}
\hfill
\begin{subfigure}{0.4\textwidth}
    \includegraphics[width=\textwidth]{RW_N1000.png}
    \caption{$N=1000$}
    \label{fig:second}
\end{subfigure}
\hfill
\begin{subfigure}{0.4\textwidth}
    \includegraphics[width=\textwidth]{RW_N10000.png}
    \caption{$N=10000$}
    \label{fig:third}
\end{subfigure}
\hfill
\begin{subfigure}{0.4\textwidth}
    \includegraphics[width=\textwidth]{RW_N20000.png}
    \caption{$N=20000$}
    \label{fig:fourth}
\end{subfigure}
\caption{In these four figures the distribuiton in (number of counts of particles) of a random walk are displayed for variouse iterations steps. It is important to note, that the scale of the a-axis changes dramatically for the four figures. For all plots the positions are counted in $25$ bins.}
\label{fig:probability_disti}
\end{figure}

The three figure \ref{fig:mean} to \ref{fig:skew} show a time-dependant statistical analysis of the random walk, of which the distributions are displayed as well.
\\ Figure \ref{fig:mean} shows the developement of the mean, figure \ref{fig:var} shows how the variance developes in time, and the third plot in figure \ref{fig:skew} displays the timedependant skew.\\
It seems on first sight, that the mean as well as the skew look like a random walk of a single particle in their own right, though this is not supported by any mathematical evidence.\\
In contrast to the \ref{fig:mean} and \ref{fig:skew} the plot of the variance in figure \ref{fig:var} dispalys a linear increasing function. Suggesting that there is structure here, that can be analysed in particular with the variance. A matter of coincidence is that when considering brownian motion, descibed by the Langevin Equation one also finds a linear dependance of the mean squared displacement of the diffusion constant. Though mathematical evidence would be required to support this claim!

\begin{figure}[h!]
    \centering
    \begin{subfigure}{0.4\textwidth}
       \includegraphics[width=\textwidth]{expectationvalues.png} 
       \caption{Mean}
       \label{fig:mean}
    \end{subfigure}
    \hfill 
    \begin{subfigure}{0.4\textwidth}
        \includegraphics[width=\textwidth]{variancenvalues.png}
        \caption{Variance}
        \label{fig:var}
    \end{subfigure} 
    \hfill 
    \begin{subfigure}{0.4\textwidth}
        \includegraphics[width=\textwidth]{scewness.png}
        \caption{Scewness}
        \label{fig:skew}
    \end{subfigure}
    \caption{These three figures display how statistical variables on the y-axis develope with increasing iteration steps on the x-axis of a random walk. In figure (a) the mean values are shown; In figure (b) the developement of the variance is shown; In figure (c) the developement of the scewness can be seen.}
    \label{fig:statistics_of_RW}
\end{figure}
\pagebreak

\section{Entropy production}
To study the entropy production we conduct a set ensembles of simulations with hard spheres in a box with reflecting boundry conditions. The ensembles will be independant simulations of varying initial positions and velocity directions. We will considere three ensemble sizes \textbf{(1)}: 5, \textbf{(2)}: 25 and \textbf{(3)}: 50. For each we shall vary the number of hard spheres to be \textbf{(a)}: 5, \textbf{(b)}: 25 and \textbf{(c)}: 50. \\
The spheres shall be initiated with a certain velocity in one half of the simulation box and 


\subsection{Implentation}

Here is the slightly modified Particle class, that contains all the particles of one simulation. Mainly the vecotors of position and velocity were collected into the one variabel - thus initiation occures with a $2$ by $number$ matrix, where the latter referres to the number of particles.
\begin{lstlisting}[language=python]
class Particle:
    def __init__(
        self,
        number: int,
        mass: float | int,
        radius: float | int,
        dimensions: int = 2
    ):
        self.n: int = number
        self.m: float = float(mass)
        self.d: int = dimensions
        self.r: float = radius

        self.x: NDArray[np.float64]= np.zeros((dimensions,number))
        self.v: NDArray[np.float64]= np.zeros((dimensions,number))
\end{lstlisting}



\begin{lstlisting}[language=python]
def initiate_positions_on_grid(
        number_of_particles: int,
        box: tuple[float|int, float|int]
) -> None:
    """
    Initializes 2D positions on a Grid with even spacing.

    Parameters:
                
        n_particles:(int)
                -> number of particles

        box:(tuple[number,number]) 
                -> box size (in [nm])

    Return:

        (NDArray[float, shpae=(dimension, number of particles)])
                -> list of vectors of positons that are aranged on a grid
    """
    grid_sections = int(np.ceil(np.sqrt(number_of_particles)))+1  # find the number of colums & rows

    # even spacing
    x_spacing = box[0]/grid_sections 
    y_spacing = box[1]/grid_sections
    # makes grid coordinates
    x, y= np.meshgrid(
        np.arange(grid_sections) * x_spacing, 
        np.arange(grid_sections) * y_spacing
    )
    positions= np.array([x.flatten()[:number_of_particles]+x_spacing/2, y.flatten()[:number_of_particles]+y_spacing/2])
    print("init positions\n",positions)  
    return positions
\end{lstlisting}

\begin{lstlisting}[language=python]
def initiate_velocities(
        n_particles: int,
        dimension:int,
        velocity: float
) -> NDArray[np.float64]:
    """
    Initiating random velocities.
    """
    velocities=  (np.random.rand(dimension,n_particles) - 0.5)
    velocities= velocity/np.linalg.norm(velocities) * velocities
    print("init velocities\n", velocities)
    return velocities
\end{lstlisting}

\begin{lstlisting}[language=python]]
def reflecting_boundry_conditions(
        particles :Particles,
        box: tuple[float|int,float|int]
        ) -> None:
    """
    Reflecs particles on the boundries given by the box.

    Parameters:

        particle: (Particle)
                -> particles of the simulation

        box: (tuple[number, number])
                -> box in which the particles are kept 
    """
    for dim in range(particles.d):
            particles.v[dim, :] = np.where(
                    particles.x[dim, :]-particles.r <0,
                    -particles.v[dim, :],
                    particles.v[dim, :]
                )
            particles.v[dim, :] = np.where(
                    particles.x[dim, :] + particles.r> box[1],
                    -particles.v[dim, :],
                    particles.v[dim, :]
                )
    pass
\end{lstlisting}

\begin{lstlisting}[language=python]
def iterate(
        particles: Particles,
        box: tuple,
        dt:float
):
    """
    Tterates the particles with elastic collision.
    """
    for i in range(particles.n):
        reflecting_boundry_conditions(particles=particles, box=box) 

        for j in range(i,particles.n):
            r_rel= particles.x[:,i]-particles.x[:,j]
            d_rel= np.linalg.norm(r_rel)
            
            if d_rel-particles.r <= 0 and d_rel!=0:
                r_pro= r_rel/d_rel
                dv= np.dot((particles.v[:,i]-particles.v[:,j]),r_pro) * r_pro
                particles.v[:,i] -= dv
                particles.v[:,j] += dv
        particles.x+= particles.v*dt  
\end{lstlisting}

\begin{lstlisting}[language=python]
    def iterate(
        particles: Particles,
        box: tuple,
        dt:float
):
    """
    Tterates the particles with elastic collision.
    """
    for i in range(particles.n):
        reflecting_boundry_conditions(particles=particles, box=box) 

        for j in range(i,particles.n):
            r_rel= particles.x[:,i]-particles.x[:,j]
            d_rel= np.linalg.norm(r_rel)
            
            if d_rel-particles.r <= 0 and d_rel!=0:
                r_pro= r_rel/d_rel
                dv= np.dot((particles.v[:,i]-particles.v[:,j]),r_pro) * r_pro
                particles.v[:,i] -= dv
                particles.v[:,j] += dv
        particles.x+= particles.v*dt  
\end{lstlisting}


\begin{lstlisting}[language=python]
def simulate_ensemble(
        num_particle: int,
        box: tuple,
        dt: float,
        time_steps: int,
        mass: float,
        velocity: float,
        radius: float,
        ensemble_size: int,
)-> NDArray[np.float32]:
    """
    Simulates an ensemble of simulations of the same kind in a for loop.

    Return: 

        data: (NDArray[float, shape=(ensemble_size, time_steps, dims, num_particle)])
    """
    ensemble: list[Particles]= [Particles(number=num_particle,mass=mass,radius=radius,dimensions=len(box)) for i in range(ensemble_size)]   
    print("Ensemble size:", len(ensemble))
    e=0
    data= np.zeros(((ensemble_size, time_steps, len(box), num_particle)))
    for p in ensemble:
        print("\n\n")
        p.x= initiate_positions_on_grid(number_of_particles=p.n,box=(box[0]/2,box[1]))
        p.v= initiate_velocities(n_particles=p.n, dimension=p.d, velocity=velocity)
        simulate(particles=p,box=box, Time_steps=time_steps, dt=dt, data=data[e])
        e+=1
    return data
\end{lstlisting}

\end{document}