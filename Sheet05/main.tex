\documentclass{article}[a4paper]

\usepackage[english]{babel}

\usepackage{subcaption}
\usepackage{graphicx}

\usepackage{amsmath}
\usepackage{amsthm}
\usepackage{amsfonts}
\usepackage{tikz}
\usepackage[plain]{algorithm}
\usepackage{algpseudocode}

\usepackage[plain]{algorithm}
% For code snippets
\usepackage{listings}
\usepackage{color}

% Define colors for code
\definecolor{codegray}{rgb}{0.5,0.5,0.5}
\definecolor{codepurple}{rgb}{0.58,0,0.82}
\definecolor{backcolour}{rgb}{0.95,0.95,0.92}


\lstdefinestyle{mystyle}{
backgroundcolor=\color{backcolour},
commentstyle=\color{codegray},
keywordstyle=\color{blue},
numberstyle=\tiny\color{codegray},
stringstyle=\color{codepurple},
basicstyle=\ttfamily\footnotesize,
breakatwhitespace=false,         
breaklines=true,                 
captionpos=b,                    
keepspaces=true,                 
numbers=left,                    
numbersep=5pt,                  
showspaces=false,                
showstringspaces=false,
showtabs=false,                  
tabsize=4
}

\lstset{style=mystyle}


\title{
    \vspace{2in}
    \textmd{\textbf{Homework: Random Walk and Entropy}}\\
    \normalsize\vspace{0.1in}\small{Due on Friday 22.11 at 10:15am}\\
    \vspace{0.1in}
    \vspace{3in}
}

\author{Dominic Nieder, Awais Ahmed}
\date{\today}


\begin{document}

\section{Introduction}

In this exercise we study how a gaussian distribution follows from a simple one-dimensional random walk

\section{Random Walk}
\subsection{Implentation \& Results}

The Implentation of the random walk is simple and quick to implement. 
We take $N=20000$ as the total number of iteration steps and $number\_of\_particles=10000$. The positions get initaited with $x_i=0$ for all particles $i$. The variable $x$ contains all the positions for all iteration steps. There is also the variable $n_bins$ which is defined for the histogramms \ref{fig:probability_disti}. \\
Within the for-loop we generate an array of uniformly sampled random numbers in the variable $dw$, which get conditioned in the next step with the \texttt{np.where}-function and replaced with $-1$ or $1$ respectivly. Last but not least the the positions get updated with in respect to the previous positions and the randomly sampled numbers. T

\begin{lstlisting}
N=20000 
n_bins=60
number_of_particles=10000  
x = np.zeros((N+1,number_of_particles),dtype=int) 
for i in range(N):
    dw= np.random.uniform(low=0.0,high=1.0,size=number_of_particles)  
    dw = np.where(dw<0.5, -1,1)  
    x[i+1,:]= x[i]+dw  # then the position get iterated
\end{lstlisting}


\begin{figure}
\centering
\begin{subfigure}{0.4\textwidth}
    \includegraphics[width=\textwidth]{RW_N100.png}
    \caption{$N=100$}
    \label{fig:first}
\end{subfigure}
\hfill
\begin{subfigure}{0.4\textwidth}
    \includegraphics[width=\textwidth]{RW_N1000.png}
    \caption{$N=1000$}
    \label{fig:second}
\end{subfigure}
\hfill
\begin{subfigure}{0.4\textwidth}
    \includegraphics[width=\textwidth]{RW_N10000.png}
    \caption{$N=10000$}
    \label{fig:third}
\end{subfigure}
\hfill
\begin{subfigure}{0.4\textwidth}
    \includegraphics[width=\textwidth]{RW_N20000.png}
    \caption{$N=20000$}
\end{subfigure}
\caption{In these four figures the distribuiton in (number of counts of particles) of a random walk are displayed for variouse iterations steps. It is important to note, that the scale of the a-axis changes dramatically for the four figures.}
\label{fig:probability_disti}
\end{figure}

\begin{figure}
    \centering
    \begin{subfigure}{0.4\textwidth}
       \includegraphics[width=\textwidth]{expectationvalues.png} 
       \caption{Mean}
    \end{subfigure}
    \hfill 
    \begin{subfigure}{0.4\textwidth}
        \includegraphics[width=\textwidth]{variancenvalues.png}
        \caption{Variance}
    \end{subfigure} 
    \hfill 
    \begin{subfigure}{0.4\textwidth}
        \includegraphics[width=\textwidth]{scewness.png}
        \caption{Scewness}
    \end{subfigure}
    \caption{These three figures display how statistical variables on the y-axis develope with increasing iteration steps on the x-axis of a random walk. In figure (a) the mean values are shown; In figure (b) the developement of the variance is shown; In figure (c) the developement of the scewness can be seen.}
\end{figure}


\end{document}