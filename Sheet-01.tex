\documentclass{article}


\usepackage{amsmath}
\usepackage{amsthm}
\usepackage{amsfonts}
\usepackage{tikz}
\usepackage[plain]{algorithm}
\usepackage{algpseudocode}

\title{
    \vspace{2in}
    \textmd{\textbf{Homework: Hard Spheres}}\\
    \normalsize\vspace{0.1in}\small{Due on Friday 25.10 at 10:15am}\\
    \vspace{0.1in}
    \vspace{3in}
}

\author{Dominic Nieder, Awais Ahmed}
\date{\today}


\begin{document}
\maketitle

\newpage

\section{Comments and introduction to our Homework}
We had some issues with the $\prime$-symbole in the eqautions, they don't seem to realize that there is a vector-arrow share the same space, and we couldn't make time with the many exercises of variouse modules take this problem seriously. If you have come across a simular issue and know how to fix it, we would appriciate to hear of it in the feedback. \\
Furthermore we are using this homework to improve and practice writing skills (as long as we find the time for it) and to give the reader a good understanding of our work and our line of thought. So feedback along those lines would also be appriciated.  \\

For this homework, where the collision had to be derived we thought for a rigorous description of what we did, the physical model needed to be described and derived. Thus we start with a Physical Model in Chapter 1, then we shortly describe how the implement the collisions 

\section{Physical Model}

    The physical model of the phere interactions is the elastic collision. The elastic collision can be described by the conservation of momentum. This follows from Newtons second law of motion in a closed system.
    \begin{align*}
        \frac{\partial \vec{P}_{tot}}{\partial t} &= \vec{F}_{ext}=0  \\
        \Leftrightarrow \vec{P}_{tot}&=\text{const} 
    \end{align*}
    Here we have $\vec{P}_{tot}=\sum_{n=1}^{N}\vec{p}_n$ of an $N$ particle system. It follows that for $\vec{P}_{tot}$ before a collision and $\vec{P}_{tot}$ after the collision $\vec{P}_{tot}=\vec{P}'_{tot}$ are equal. \\

    Now lets considere two particels with a pair of phase space coordinates $(\vec{r}_1, \vec{p}_1)$ and $(\vec{r}_2, \vec{p}_2)$ resulting in a momentum $\vec{P}_{tot}=\vec{p}_1+\vec{p}_2$ of the closed system. Then due to the conservation of momentum 
    \begin{align*}
        \vec{P}_{tot}&=\vec{P}'_{tot} \\
        \Leftrightarrow \vec{p}_1+\vec{p}_2&= \vec{p}_1'+\vec{p}_2' \\
        \Leftrightarrow \vec{p}_1 - \vec{p}_1' &= - (\vec{p}_2 - \vec{p}_2') \\
        \Leftrightarrow \Delta \vec{p}_1 &= -\Delta\vec{p}_2
    \end{align*}
    The problem is solved when the momenta $\vec{p}'_1$ and $\vec{p}_2'$ after the collsision have been found.\\ % Primes are ugly af

    A possibility to find the momenta after the collision can be by changing into the relative coordinates $\vec{r}_{rel}=\vec{r}_2-\vec{r}_1$ of the system. 
    The resulting relative momentum can be derived while considering $\frac{\partial \vec{r}}{\partial t}=\frac{\vec{p}}{m}$ 
    \begin{align*}
        \frac{\partial \vec{r}_{rel}}{\partial t}&= \frac{\partial \vec{r}_2}{\partial t}-\frac{\partial \vec{r}_1}{\partial t} \\
        &=\frac{\vec{p}_2}{m_2}-\frac{\vec{p}_1}{m_1} 
    \end{align*}
    Here we can use $\vec{P}_{tot}-\vec{p}_1=\vec{p}_2$:
    \begin{align*}
        \frac{\partial \vec{r}_{rel}}{\partial t} &= \frac{\vec{P}_{tot}-\vec{p}_1}{m_2} - \frac{\vec{p}_1}{m_1} \\
        &= \frac{\vec{P}_{tot}}{m_2} - \left(\frac{\vec{p}_1}{m_2}+\frac{\vec{p}_1}{m_1}\right) \\
        &= \frac{\vec{P}_{tot}}{m_2} - \underset{=\mu^{-1}}{\underbrace{\left(\frac{m_1+m_2}{m_2m_1}\right)}}  \vec{p}_1 \\
        \Leftrightarrow \mu \frac{\partial \vec{r}_{rel}}{\partial t}= \vec{p}_{rel} &= \frac{\mu}{m_2} \vec{P}_{tot} - \vec{p}_1
    \end{align*} 
    Now, we want to considere the change in momenta (in respect to time) taking advantage of $\frac{\partial \vec{P}_{tot}}{\partial t}=0$. We obtain
    \begin{align*}
        \frac{\partial \vec{p}_{rel}}{\partial t} =-\frac{\partial \vec{p}_1}{\partial t} \\
        \Leftrightarrow \Delta \vec{p}_{rel} =-\Delta \vec{p}_1 \\
        \Rightarrow \Delta \vec{p}_2 = \Delta\vec{p}_{rel}.
    \end{align*}
    The advantage of relative koordinates is the translation into the koordinates of one of the solid spheres i.e. of $\vec{r}_2$ and describing the kollision of a moving sphere and a resting sphere. \\
    
    It remains an open problem to describe the change in momenta in the elastic collision of a resting hard sphere and a sphere moving. Here it is convinient to make use of the simplification, that the hard spheres have the same mass $m_i=m_j$ for all $i$ and $j$, or in a two body collision $m_1=m_2$. \\

    For a two body collision, for which the mass of the bodies are the same 

\section{Implementation}

The collisions get 

\section{Simulation and Analysis}
\end{document}