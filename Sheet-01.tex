\documentclass{article}


\usepackage{amsmath}
\usepackage{amsthm}
\usepackage{amsfonts}
\usepackage{tikz}
\usepackage[plain]{algorithm}
\usepackage{algpseudocode}

\title{
    \vspace{2in}
    \textmd{\textbf{Homework: Hard Spheres}}\\
    \normalsize\vspace{0.1in}\small{Due on Friday 25.10 at 10:15am}\\
    \vspace{0.1in}
    \vspace{3in}
}

\author{Dominic Nieder, Awais Ahmed}
\date{\today}


\begin{document}
\maketitle

\newpage

\section{Physical Model}
    The physical model of the phere interactions is the elastic collision. The elastic collision can be described by the conservation of momentum. This follows from Newtons second law of motion in a closed system.
    \begin{align*}
        \frac{\partial \vec{P}_{tot}}{\partial t} &= \vec{F}_{ext}=0  \\
        \Leftrightarrow \vec{P}_{tot}&=\text{const} 
    \end{align*}
    Here we have $\vec{P}_{tot}=\sum_{n=1}^{N}\vec{p}_n$ of an $N$ particle system. It follows that for $\vec{P}_{tot}$ before a collision and $\vec{P}_{tot}$ after the collision $\vec{P}_{tot}=\vec{P}'_{tot}$ are equal. \\
    Now lets considere two particels with a pair of phase space coordinates $(\vec{r}_1, \vec{p}_1)$ and $(\vec{r}_2, \vec{p}_2)$ resulting in a momentum $\vec{P}_{tot}=\vec{p}_1+\vec{p}_2$ of the closed system. Then due to the conservation of momentum 
    \begin{align*}
        \vec{P}_{tot}&=\vec{P}'_{tot} \\
        \Leftrightarrow \vec{p}_1+\vec{p}_2&= \vec{p}_1'+\vec{p}_2' \\
        \Leftrightarrow \vec{p}_1 - \vec{p}_1' &= - (\vec{p}_2 - \vec{p}_2') \\
        \Leftrightarrow \Delta \vec{p}_1 &= -\Delta\vec{p}_2
    \end{align*}
    The problem is solved when the momenta $\vec{p}'_1$ and $\vec{p}_2'$ after the collsision have been found.\\ % Primes are ugly af
    A possibility to find the momenta after the collision can be by changing into the relative coordinates $\vec{r}_{rel}=\vec{r}_2-\vec{r}_1$ of the system. 
    The resulting relative momentum can be derived while considering $\frac{\partial \vec{r}}{\partial t}=\frac{\vec{p}}{m}$ 
    \begin{align*}
        \frac{\partial \vec{r}_{rel}}{\partial t}&= \frac{\partial \vec{r}_2}{\partial t}-\frac{\partial \vec{r}_1}{\partial t} \\
        &=\frac{\vec{p}_2}{m_2}-\frac{\vec{p}_1}{m_1} 
    \end{align*}
    Here we can use $\vec{P}_{tot}-\vec{p}_1=\vec{p}_2$:
    \begin{align*}
        \frac{\partial \vec{r}_{rel}}{\partial t} &= \frac{\vec{P}_{tot}-\vec{p}_1}{m_2} - \frac{\vec{p}_1}{m_1} \\
        &= \frac{\vec{P}_{tot}}{m_2} - \left(\frac{\vec{p}_1}{m_2}+\frac{\vec{p}_1}{m_1}\right) \\
        &= \frac{\vec{P}_{tot}}{m_2} - \underset{=\mu^{-1}}{\underbrace{\left(\frac{m_1+m_2}{m_2m_1}\right)}}  \vec{p}_1 \\
        \Leftrightarrow \mu \frac{\partial \vec{r}_{rel}}{\partial t}= \vec{p}_{rel} &= \mu \frac{\vec{P}_{tot}}{m_2} - \vec{p}_1
    \end{align*} 
    Now, considering the change in momenta in respect to time and reminding of ourselves that $\frac{\partial \vec{P}_{tot}}{\partial t}=0$ we get
    \begin{equation*}
        \frac{\partial \vec{p}_{rel}}{\partial t} =-\frac{\partial \vec{p}_1}{\partial t}.
    \end{equation*}
    The advantage of relative koordinates is the translation into the koordinates of one of the solid spheres i.e. of $\vec{r}_2$ and describing the kollision of a moving sphere and a resting sphere.

\section{Implementation}
    

\section{Simulation}


\section{Analysis of Velocity distributions}


\end{document}