\documentclass{article}[a4paper]

\usepackage[english]{babel}

\usepackage{subcaption}
\usepackage{graphicx}

\usepackage{amsmath}
\usepackage{amsthm}
\usepackage{amsfonts}
\usepackage{tikz}
\usepackage[plain]{algorithm}
\usepackage{algpseudocode}

\usepackage[plain]{algorithm}
% For code snippets
\usepackage{listings}
\usepackage{color}

% Define colors for code
\definecolor{codegray}{rgb}{0.5,0.5,0.5}
\definecolor{codepurple}{rgb}{0.58,0,0.82}
\definecolor{backcolour}{rgb}{0.95,0.95,0.92}


\lstdefinestyle{mystyle}{
backgroundcolor=\color{backcolour},
commentstyle=\color{codegray},
keywordstyle=\color{blue},
numberstyle=\tiny\color{codegray},
stringstyle=\color{codepurple},
basicstyle=\ttfamily\footnotesize,
breakatwhitespace=false,         
breaklines=true,                 
captionpos=b,                    
keepspaces=true,                 
numbers=left,                    
numbersep=5pt,                  
showspaces=false,                
showstringspaces=false,
showtabs=false,                  
tabsize=4
}

\lstset{style=mystyle}


\title{
    \vspace{2in}
    \textmd{\textbf{Homework: RDF and freee energies}}\\
    \normalsize\vspace{0.1in}\small{Due on Friday 22.11 at 10:15am}\\
    \vspace{0.1in}
    \vspace{3in}
}

\author{Dominic Nieder, Awais Ahmed}
\date{\today}

\begin{document}
\section{Introduction}
In this exercise it is the task to study the radial distribution function as well as the potential of mean force by example of a hard sphere simulation. These two methods of statistical analysis help visualize statistical information that is hidden by conventional mean e.g. an animation or looking at the energy landscapes. They can capture the forces that emerge though the complexity of the system itselfe.

First we shall take a look at the radial distribution funciton then we hope to take a look at the potential of mean force.

\section{Radial distribution fuction}
\subsection{Imlementation}

The \texttt{Particle} class again contains all variables. 
Since the last sheet, miniscule changes have been introduced i.e. \textit{self.box} which contains all corners of the box, which will be used for distance calculations between particles. 
Then additionally, the particles shall also get initiated to the initial positions (regular grid) and velocities (randomly) get initiated directly and get displayed.
\begin{lstlisting}[language=Python]
class Particles:
    def __init__(
        self,
        number: int,
        mass: float | int,
        velocity: float,
        radius: float | int,
        box: tuple[float|int, float|int]
    ):
        """
        Initiate particles with unpecified positions, velocities, accelerations.
        """
        self.n: int = number
        self.m: float = float(mass)  #  kg
        self.d: int = len(box)
        self.r: float = radius  # nm
        self.box: NDArray[np.float32]= np.array([[0,box[0],0,box[0]],[0,0,box[1],box[1]]])  # nm
        print("ititated",self.n, "particles:\n\tradius",self.r,"\tmass",self.m,"\tdimensions",self.d)
        self.x: NDArray[np.float32]= initiate_positions_on_grid(number,box) 
        self.v: NDArray[np.float32]= initiate_velocities(number, self.d, velocity)
        fig = plt.figure()
        ax = fig.add_subplot()
        ax.set_title("Initiation: Positions")
        ax.scatter(self.x[0,:], self.x[1,:])
        ax.grid(True)
        plt.show()      
        plt.close()
\end{lstlisting}

Then the \texttt{relative\_distances} function was implemented, so that the nearest distance between two particles gets calculated. 
For this no clever implementation was used, but rather a simple and straight forward one. 
In which all possible positions get calculated and the smalles one gets passed together with its norm.
\begin{lstlisting}[language=Python]
def relative_distnace(
        x_1: NDArray,
        x_2: NDArray,
        box: NDArray
) -> NDArray:
    rel6 = np.zeros((len(box[:,0]),7))
    rel6[:,0] = x_1-x_2 
    rel6[:,1] = x_1-x_2 + box[:,1]
    rel6[:,2] = x_1-x_2 + box[:,2]
    rel6[:,3] = x_1-x_2 + box[:,3]
    rel6[:,4] = x_1-x_2 - box[:,1]
    rel6[:,5] = x_1-x_2 - box[:,2]
    rel6[:,6] = x_1-x_2 - box[:,3]
    d6 = np.linalg.vector_norm(rel6, axis=0)
    min = np.argmin(d6)
    return rel6[:,min], d6[min]
\end{lstlisting}

To determine the kinematics of two colliding particles, the function \texttt{hard\_sphere\_collision} was implemented. 
This function iterates through each pairs of particles $i$ and $j$ determines the relative distance vector and its' norm with \texttt{relative\_distance}. 
The collision condition is met when the distance is smaller than twice the radius and its kinemativs gets determined as on the priviouse task sheet. 
\begin{lstlisting}[language=Python]
def hard_sphere_collision(
        particles:Particles,
):
    for i in range(particles.n):
        for j in range(i+1,particles.n):
            r_rel, d_rel= relative_distnace(particles.x[:,i], particles.x[:,j],particles.box) 
            if d_rel-particles.r <= 0:
                # print("collision detected!")
                r_pro= r_rel/d_rel
                dv= np.dot((particles.v[:,i]-particles.v[:,j]),r_pro) * r_pro
                particles.v[:,i] -= dv
                particles.v[:,j] += dv
\end{lstlisting}

The iterate function calls the \texttt{hard\_sphere\_collision} function and iterates all positions and makes sure the particles are contained by their periodic boundry conditions. The function \texttt{simulate} calls this function then to run the simulation, this function returns \textit{data} of the simulation run.
\begin{lstlisting}[language=Python]
def iterate(
        particles: Particles,
        dt:float
):
    hard_sphere_collision(particles=particles)
    particles.x+= particles.v*dt  
    particles.x = np.mod(particles.x, [[particles.box[0,3]],[particles.box[1,3]]]) 
\end{lstlisting}

Coming to the radial distribution function \texttt{rdf}. 
It takes \textit{particles}, \textit{data} and \textit{n\_bins} as arguments. 
The \textit{data} comes from the simulation and \textit{n\_bins} gives the area between $r, r+dr$. 
This function has three for loops. 
One, for all time steps. 
Two for each particle and the third through all remaining particles.
In the two forloops the distances between all particles gets determined from which a histogram gets created. 
Then because this historgram consideres the mean of $n$ particles we devide by \textit{particles.n} and also by \textit{T} as we take the timewise mean as well. 
We choose the bounds of the histogram by taking the distance from the center of the box to one corner $\sqrt((\textit{box[0,3]}/2)^2+(\textit{box[1,3]}/2)^2)$, which should be the maximal distance between two particles in a box under periodic boundry conditions.
The minimal distance between two particles should be $2\textit{particles.r}$.
Never the less, to see the how well the simulation performs (considering overlapping spheres) we start collecting bin points at \textit{particles.r}.

\begin{lstlisting}
def rdf(
        particles: Particles,
        data: NDArray,
        n_bins: int
) -> NDArray:
    T= len(data[:,0,0])
    rho = particles.n/(particles.box[0,3]*particles.box[1,3])   # isotropic density
    rdf_hist= np.zeros((T,n_bins))
    distances= np.zeros((particles.n,particles.n))
    for t in range(T):
        rdf_histi=np.zeros(n_bins)
        for i in range(particles.n):
            for j in range(i+1,particles.n):
                distances[i,j]=relative_distnace(data[t,:,i],data[t,:,j],particles.box)[1]
                distances[j,i]=distances[i,j]
        #print(distances)
        rdf_histi, r=np.histogram(distances,n_bins,(particles.r, np.sqrt((particles.box[0,3]/2)**2+(particles.box[1,3]/2)**2)))
        rdf_hist[t] = rdf_histi
        rdf_hist=rdf_hist/particles.n/T
    return np.sum(rdf_hist,axis=0)/(rho*2*np.pi), r[:-1]
\end{lstlisting}

\subsection{Results}
Resulting from the simulation and the radial distribution function we get the following graph \ref{fig:rdf}.
It seems to be stagnating -- though increasing density of particles with increasing distance.
\begin{figure}[h!]
    \centering
    \includegraphics[width=0.6\textwidth]{rdf.png}
    \caption{This figure displays the mean radial distribution function. On the x-axis we see the relative distances. On the y-axis the distribution values. We took values up until $x=\sqrt{2(\textit{box}/2)^2}$nm as this should be the minimal possible distance of two particles in periodic boundry conditions.} \label{fig:rdf}
\end{figure}


\section{Potential of mean force}
\subsection{implementation}
The only functions needed to be implemented for the second part of the exercise were the functions to determine the temperature of the system (thus also mean kinetik energy) as well as the mean free energy (including the partition function).
The mean kinetic energy is determined by summing over all velocities and deviding by the number of particles.
\begin{lstlisting}[language=Python]
def mean_kinetik_energy(particles:Particles) -> float:
    """Determines mean kinetik energy of particles"""
    return particles.m/2*np.sum( np.linalg.vecdot( particles.v,particles.v, axis=0))/particles.n
\end{lstlisting}
Then to obtain the temperature one devides by the Boltzmann constant.
\begin{lstlisting}
def temperature(particles:Particles) -> float:
    """Determines Temperature of particles"""
    k_B= 1.38e-23  
    return mean_kinetik_energy(particles=particles)/k
\end{lstlisting}
After this we can determine the potential of mean force by the equation
\begin{equation}
    g(r)= \exp \Bigl\{-\frac{\Delta F(r)}{k_B T}\Bigr\}
\end{equation}
\begin{lstlisting}
def potential_of_maen_force(temperature:float,rdf:NDArray) -> NDArray:
    k_B= 1.38e-23  # J/K
    return -np.log(rdf)*k_B*temperature   
\end{lstlisting}

\subsection{Results}
In this figure the potential of mean force can be seen.
Although the radial distribution function seems as a mystory to me, the mean force potential could make sense: 
As there are no interactions between the particles at a distance other from $2\textit{particles.r}$ we see no potential landscape.
Only for $r$ near the boundry where the two particles would interact ($2\textit{particles.r}$).
There we see a seemingly infinit boundry, just how the interaction potential should look.
\begin{figure}
    \centering
    \includegraphics[width=0.6\textwidth]{pot_mean_force.png}
    \caption{In this figure the potential of mean force can be seen. On the x-axis the distances are given and on the y-axis the potential mean force is given in J.}
\end{figure}


\end{document}